%% LaTeX-Beamer template for KIT design
%% by Erik Burger, Christian Hammer
%% title picture by Klaus Krogmann
%%
%% version 2.1
%%
%% mostly compatible to KIT corporate design v2.0
%% http://intranet.kit.edu/gestaltungsrichtlinien.php
%%
%% Problems, bugs and comments to
%% burger@kit.edu

\documentclass[18pt]{beamer}

%% SLIDE FORMAT

% use 'beamerthemekit' for standard 4:3 ratio
% for widescreen slides (16:9), use 'beamerthemekitwide'

\usepackage{templates/beamerthemekit}
%\usepackage{templates/beamerthemekitblack}
%\usepackage{templates/beamerthemekitwide}

\usepackage{xspace}
% \usepackage{amssymb}
\usepackage{wasysym}
\usepackage{graphicx}
\usepackage{tikz}
\usepackage{listings}
\usetikzlibrary{positioning}
\usetikzlibrary{arrows}
\usetikzlibrary{decorations.pathmorphing}
\usetikzlibrary{fit}
\usetikzlibrary{shapes}
\usetikzlibrary{calc}
\usepackage[utf8]{inputenc}
\usepackage{csquotes}

\setbeamertemplate{navigation symbols}{}
\setbeamercovered{invisible}

%% TITLE PICTURE

% if a custom picture is to be used on the title page, copy it into the 'logos'
% directory, in the line below, replace 'mypicture' with the 
% filename (without extension) and uncomment the following line
% (picture proportions: 63 : 20 for standard, 169 : 40 for wide
% *.eps format if you use latex+dvips+ps2pdf, 
% *.jpg/*.png/*.pdf if you use pdflatex)

\titleimage{Castel0}

% (*.eps format if you use latex+dvips+ps2pdf,
% *.jpg/*.png/*.pdf if you use pdflatex)

%% TikZ INTEGRATION

% use these packages for PCM symbols and UML classes
% \usepackage{templates/tikzkit}
% \usepackage{templates/tikzuml}

% the presentation starts here

\newcommand{\kit}[1]{\textcolor{kit-green100}{#1}}

\title{Quiz}
\subtitle{Verteilte Systeme}
\author{Martin Hecker, Simon Greiner}

\institute{Kompetenzzentrum für angewandte Sicherheitstechnologie}

% Bibliography

%\usepackage[citestyle=authoryear,bibstyle=numeric,hyperref,backend=biber]{biblatex}
%\addbibresource{templates/example.bib}
%\bibhang1em

\newboolean{sol}
\setboolean{sol}{false}

\newcommand{\stereo}[1]{$\ll$#1$\gg$\xspace}
\newcommand{\includesstereo}[2]{\stereo{$#1$ \emph{includes} $#2$}}
\newcommand{\sicherunsicher}{%
Spezifikation erfüllt?         \hfill \Square \\
}
\newcommand{\securewrt}[2]{%
\ifsol
Spezifikation erfüllt nach #1  \hfill #2 \\
\fi
}

\begin{document}

% change the following line to "ngerman" for German style date and logos
%\selectlanguage{english}
\selectlanguage{ngerman}


\AtBeginSection{\frame{\sectionpage}}

\begin{frame}
\titlepage
\end{frame}

\include{examples-include}

\begin{frame}[fragile,t]{Interpretationen}

\begin{overprint}
\onslide<1>
A component satisfies its specification if for every dataset $d$, output defined (by $d$) to be {\color{blue} low} is not influenced
by input defined (by $d$) to be {\color{red} high} .
\onslide<2>
Insbesondere ist eine Spezifikation genau dann erfüllt wenn für alle Eingabeparameter $i$, und alle Ausgabeparameter $o$ gilt:
\onslide<3>
Intuitiv bedeutet eine Annotation \includesstereo{ds}{\ldots, p \ldots} also:
\end{overprint}

\vfill

\kit{Interpretation i}
\begin{overprint}
\onslide<1>
For a given dataset $d$ and stereotype \includesstereo{ds}{P},
$d$ defines parameters $p \in P$ to be {\color{red} high},  all others to be  {\color{blue} low}.
\onslide<2>
\begin{itemize}
  \item Falls $o$ durch $i$ beeinflusst wird, dann  $DS_i \subseteq DS_o$
\end{itemize}
\onslide<3>
\begin{itemize}
  \item Es ist möglich,    dass ein Teil der Information die \enquote{in $p$ steckt}, aus Dataset $ds$ ist.
\end{itemize}
\end{overprint}

\vfill

\kit{Interpretation ii}
\begin{overprint}
\onslide<1>
For a given dataset $d$, and stereotype \includesstereo{d}{P},
$d$ defines parameters $p \in P$ to be {\color{blue} low},  all others to be  {\color{red} high}.
\onslide<2>
\begin{itemize}
  \item Falls $o$ durch $i$ beeinflusst wird, dann  $DS_o \subseteq DS_i$
\end{itemize}
\onslide<3>
\begin{itemize}
   \item Es ist garantiert, dass alle Information         die \enquote{in $p$ steckt}, aus Dataset $ds$ ist\footnote{lasst Euch nicht von Simon erzählen, dass Interpretation ii irgendeine andere intuitive Interpretation hat! :)}.
\end{itemize}
\end{overprint}

\vfill

\begin{overprint}
\onslide<2>
wobei $DS_i$ die Menge der Datasets ist, in denen $i$ included ist, und
      $DS_o$ die Menge der Datasets ist, in denen $o$ included ist.
\end{overprint}


\end{frame}
\appendix
\beginbackup

%\begin{frame}[allowframebreaks]{References}
%\printbibliography
%\end{frame}

\backupend

\end{document}
