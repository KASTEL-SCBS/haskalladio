\textbf{Hilfsbegriffe 3/n} (\texttt{AbstractDesignModel} in \texttt{Security.hs})\\



Relationen:
\begin{description}
  \item[$\tobeinferred{linksPayloadFullyAccessibleBy(Attackers, LinkingResource)}$]
        wo $\tobeinferred{linksPayloadFullyAccessibleBy(a, link)}$ gdw. es Angreifer a
        möglich ist, vollen Zugriff auf \enquote{Nutzdaten} zu gelangem, die 
        über $link$ übertragen werden:

\[
   \inferrule{\tobeinferred{linksPhysicalAccessibleBy(a,link)} \\
              \tobeinferred{exposesPhsicallyAccessiblePayloadTo(link, a))} 
             }
             {\tobeinferred{linksPayloadFullyAccessibleBy(a, link)}}
\]


  \item[$\tobeinferred{linksMetaDataFullyAccessibleBy(Attackers, LinkingResource)}$]
        wo $\tobeinferred{linksMetaDataFullyAccessibleBy(a, link)}$ gdw. es Angreifer a
        möglich ist, vollen Zugriff auf \enquote{Nutzdaten} zu gelangem, die 
        über $link$ übertragen werden:

\[
   \inferrule{\tobeinferred{linksPhysicalAccessibleBy(a,link)} \\
              \tobeinferred{exposesPhsicallyAccessibleMetaDataTo(link, a))} 
             }
             {\tobeinferred{linksMetaDataFullyAccessibleBy(a, link)}}
\]

\end{description}

\todo{Konzept Payload vs MetaData auch in Service-Parameter-Spezifikation integrieren} 
