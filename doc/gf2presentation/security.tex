\documentclass[varwidth=15cm]{standalone}
\usepackage{rules}

\begin{document}
\textbf{Elementare Sicherheits-Modellierung} (\texttt{BasicDesignModel} in \texttt{Security.hs})\\

Mengen: $\tobespecified{Attackers}$ \discuss{Eigentlich eher: $Benutzer$ oder $Benutzerrolle$}, \\
        $\tobespecified{DataSets}$ \\

$DataSets:$ Datenklassen, z.B. \enquote{Anwender} oder \enquote{Stromverbrauch} oder \enquote{Öffentlich} \\


Relationen:
\begin{description}
  \item[$\tobespecified{classificationOf(Parameter, DataSets)}$]
        mit $\tobespecified{classificationOf}(p, ds)$ gdw. Parameter $p$ als $ds$ klassifiziert ist
  \item[$\tobespecified{classificationOfCall(Service, DataSets)}$]
        mit $\tobespecified{classificationOf}(s, ds)$ gdw. wenn die Information, dass
        Service $s$ aufgerufen wurde, als $ds$ klassifiziert ist.
  \item[$\tobespecified{dataAllowedToBeAccessed(Attacker, DataSets)}$]
        mit $\tobespecified{dataAllowedToBeAccessed}(a, ds)$ gdw. es dem Angreifer $a$ (\discuss{Benutzer!}, s.o.)
        erlaubt ist, Informationen der  Klasse $ds$ zu erlernen. \\

  \item[$\tobespecified{interfacesAllowedToBeUsedBy(Attacker, Interface)}$]
        mit $\tobespecified{interfacesAllowedToBeUsedBy}(a, i)$ gdw. dem Angreifer $a$ (\discuss{Benutzer!}, s.o.) erlaubt ist, Services vom Interface $i$ zu verwenden.\\[0.5cm]
       \discuss{z.B. kann eine Analyse schon dann meckern,
       wenn $a$ Zugriff auf $i$ hat, der Rückgabeparameter eines Services
        $s$ von $i$ aber einer Datenklasse $ds$ zugeordnet ist,
        über die $a$ nichts lernen darf. \\
        Andererseits sollte dies aber vielleicht allein über die Locations von $a$ modelliert werden?!?!}


\end{description}
\end{document}
