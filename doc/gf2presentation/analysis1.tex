\documentclass[varwidth=25cm]{standalone}
\usepackage{rules}

\begin{document}
\textbf{Hilfsbegriffe 1/n} (\texttt{AbstractDesignModel} in \texttt{Security.hs})\\



Relationen:
\begin{description}
  \item[$\tobeinferred{containersPhysicallyAccessibleBy(Attackers, ResourceContainers)}$]
        wo $\tobeinferred{containersFullyAccessibleBy(a, rc)}$ gdw. es Angreifer $a$
        möglich ist, \enquote{physikalischen}  Zugriff auf RessourceContainer $rc$ zu erlangen,
        also zugang zum \enquote{Standort} von $rc$ hat.

        Dies bedeutet noch nicht notwendigerweise, dass er auch alle Daten dort ablesen kann,
        z.B. wenn der RessourceContainer keine Schnittstellen hierzu ermöglicht,
        oder irgendwie gesichert ist.
\[
   \inferrule{\tobespecified{location(rc)}=loc \\
              \tobespecified{locationsAccessibleBy(a,loc)}
             }
             {\tobeinferred{containersPhysicalAccessibleBy(a,rc)}}
\]


  \item[$\tobeinferred{linksPhysicallyAccessibleBy(Attackers, LinkingResource)}$]
        wo $\tobeinferred{containersFullyAccessibleBy(a, rc)}$ gdw. es Angreifer $a$
        möglich ist, \enquote{physikalischen}  Zugriff auf RessourceContainer $rc$ zu erlangen,
        also zugang zum \enquote{Standort} von $rc$ hat.

\[
   \inferrule{\tobespecified{linkLocation(link,loc)} \\
              \tobespecified{locationsAccessibleBy(a,loc)}
             }
             {\tobeinferred{linksPhysicalAccessibleBy(a,rc)}}
\]

\end{description}

\end{document}
