\documentclass[varwidth=25cm]{standalone}
\usepackage{rules}

\begin{document}
\textbf{Hilfsbegriffe 2/n} (\texttt{AbstractDesignModel} in \texttt{Security.hs})\\



Relationen:
\begin{description}
  \item[$\tobeinferred{containersFullyAccessibleBy(Attackers, ResourceContainers)}$]
        wo $\tobeinferred{containersFullyAccessibleBy(a, rc)}$ gdw. es Angreifer a
        möglich ist, vollen Zugriff zu gelangen auf alle Informationen die 
        auf dem RessourceContainer $rc$ zu finden sind:

\[
   \inferrule{\tobeinferred{containersPhysicalAccessibleBy(a,rc)} \\
              \tobespecified{containerTamperableByAttackerWithAbilities(c, tamperingAbilities(a))} 
             }
             {\tobeinferred{containersFullyAccessibleBy(a, rc)}}
\]

\[
   \inferrule{\tobeinferred{containersPhysicalAccessibleBy(a,rc)} \\
              \tobespecified{sharing(rc)=openShared} \\
              \tobespecified{furtherConnections(rc)=possible}
             }
             {\tobeinferred{containersFullyAccessibleBy(a, rc)}}
\]

\[
   \inferrule{\tobespecified{sharing(rc)=openShared} \\
              \tobespecified{furtherConnections(rc)=existing}
             }
             {\tobeinferred{containersFullyAccessibleBy(a, rc)}}
\]
\end{description}


\discuss{Ist in der zweiten Regel die Bedingung $\tobespecified{sharing(container)=openShared}$
wirklich gerechtfertigt? Kann $a$ nicht selbst dann durch tampern an die Daten ran, wenn $rc$ $controlledExclusive$ ist? 
}
\\[0.5cm]
\discuss{Letzte Regel ist recht pessimistisch: hier wird \emph{jedem} Angreifer zugetraut,
dass er irgendwie ein Böses Programm (z.B. ne App) auf $rc$ installiert bekommt.}

\end{document}
