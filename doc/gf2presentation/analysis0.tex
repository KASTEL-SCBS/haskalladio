\documentclass[varwidth=25cm]{standalone}
\usepackage{rules}

\begin{document}
\textbf{Hilfsbegriffe 0/n} (\texttt{LinkAccessModel} in \texttt{Security.hs})\\

In beiden Varianten interessiert uns folgende \tobeinferred{abgeleitete} Begriffe:

Relationen:
\begin{description}
  \item[$\tobeinferred{exposesPhsicallyAccessiblePayloadTo(LinkingResources, Attackerss)}$]
        wo $\tobeinferred{exposesPhsicallyAccessiblePayloadTo(link, a)}$
        gdw. $a$ kann \enquote{Nutzdaten} auf $link$ lesen, falls er \enquote{physischen} Zugriff
        auf das Medium erlangt.

  \item[$\tobeinferred{exposesPhsicallyAccessibleMetaDataTo(LinkingResources, Attackerss)}$]
        wo $\tobeinferred{exposesPhsicallyAccessiblePayloadTo(link, a)}$
        gdw. $a$ kann \enquote{Meta Daten} auf $link$ lesen, falls er \enquote{physischen} Zugriff
        auf das Medium erlangt.
  
  \item[Für das Simple Modell:]
\[
   \inferrule{\neg \tobespecified{isEncrypted(link)} \\
             }
             {\tobeinferred{exposesPhsicallyAccessiblePayloadTo(link,a)}}
\]
\[
   \inferrule{\ }
             {\tobeinferred{exposesPhsicallyAccessibleMetaData(link,a)}}
\]

\end{description}

\end{document}
